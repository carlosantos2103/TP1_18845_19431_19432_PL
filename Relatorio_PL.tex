\documentclass{article}
\usepackage[T1]{fontenc}
\usepackage[utf8]{inputenc}
\usepackage[portuges]{babel}
\usepackage{graphicx}
\begin{document}
\centerline{Instituto Politécnico Cávado do Ave \linebreak}
\vspace*{1 em}
\centerline{Curso engenharia de Sistemas Informáticos \linebreak}
\vspace*{1 em}
\centerline{Processamento de Linguagem \linebreak}
\vspace*{5 em}

\centerline{Autores}

\vspace*{3 em}
\centerline{Carlos Santos Nº 19432\linebreak}
\vspace*{1 em}
\centerline{João Rodrigues Nº 19431\linebreak}
\vspace*{1 em}
\centerline{João Ricardo Nº 18845 \linebreak}
\vspace*{3 em}

\centerline{\textbf{Test Anything Protocol}}

\vspace*{10 em}
\centerline{Data: /11/2020}

\newpage
%% ----------- Pagina 2
\centerline{\textbf{ÍNDICE}}
\vspace*{3 em}

\begin{enumerate}
	\item Resumo \hspace{30 em} 2 \\
	\item Contextualização \hspace{26 em} 3\\
	\item Conclusão \hspace{29 em} ? \\
	\item Bibliografia\hspace{28,75 em} ? \\
\end{enumerate}
%% ----------- Pagina 3

\newpage
\centerline{\textbf{RESUMO}}

\vspace*{3 em}
O \textit{Test Anything Protocol} é um formato textual usado por ferramentasde testes unitários desenvolvidas
para várias linguagens, desde o Perl ao C. \linebreak
\vspace*{1 em}

\begin{figure}[!htb]
	\centering
	\includegraphics[scale=0.9]{figura2}
	%%\caption{Legenda da Imagem}
	%%\label{Label de referência para a imagem}
\end{figure}
\vspace*{2 em}
O que se pretende é o desenvolvimento de uma ferramenta que permita analisar este tipo de outputs.
\vspace*{2 em}

Pretende-se que tenha as seguintes funcionalidades:

\vspace*{1 em}

\begin{itemize}
\item Gerar, um relatório para cada ficheiro, que contenha um resumo de números de testes executados, número de testes com resultado positivo e percentagem de falhas;
\item Gerar um HTML para cada um dos relatório, e que o apresente de forma visual (e colorida) quais os testes com sucesso;
\item Dada uma pasta com vários ficheiros, em que cada um é um relatório independente, gerar páginas
HTML interligadas em que é possível consultar visualmente o dados dos relatórios.
\end{itemize}

\footnote[1]{https://testanything.org/}

\newpage

% ------- Pagina 4 ------

\centerline{\textbf{Contextualização}}
\vspace{3 em}
Este documento foi realizado no contexto da unidade curricular Processamento de Linguagem, do Instituto Politécnico Cávado do Ave.
Neste trabalho pretende-se criar uma ferramenta de análise dos ficheiros.

\vspace{3 em}
O modo da abordagem deste tipo de problema foi: 
\vspace{1 em}
\begin{itemize}
	\item Criar expressões regulares capazes de ler os ficheiros;
	\item Guardar os Dados de forma a serem manipulados mais tarde;
	\item Acrescentar mais tarde...
\end{itemize}

\newpage

% ------- Pagina 5 -------
\centerline{\textbf{Criação de Expressões Regulares}}

\vspace{3 em}

Uma Expressão Regular é uma forma concisa e flexível de identificar cadeias de caracteres de interesse, como caracteres particulares, palavras ou padrões de caracteres. Estas Expressões Regulares são analisadas e um processador de expressão regulares examinam os textos e identificam as partes que coincidem. \\

\vspace{1,5 em}
Para a escolha das expressões foram feitas análises e sucessivas tentativas a ficheiros de testes. \\

\begin{figure}[!htb]
	\centering
	\includegraphics[scale=0.9]{figura3}
	\caption{Exemplo de um Ficheiro de Teste - teste3.t}
	%%\label{Label de referência para a imagem}
\end{figure}

\vspace{3 em}


\newpage
% -------- Pagina 6 ---------
\centerline{\textbf{Armazenamento de Dados}}

\vspace{3 em}

Os Dados para serem manipulados têm de ser guardados em estruturas de dados eficientes, tendo em conta que será necessário fazer cálculos estatísticos. Assim sendo os valores estarão guardados numa Classe  \\

\vspace{1,5 em}
\begin{figure}[!htb]
	\centering
	\includegraphics[scale=0.8]{figura4}
	\caption{Exemplo da Classe Usada}
	%%\label{Label de referência para a imagem}
\end{figure}

\vspace{3 em}
Nesta classe são guardados os dados referentes aos ficheiros lidos.

\vspace{2 em}
\begin{itemize}
	\item resultado - Ok ou Not Ok;
	\item numero - Número do teste;
	\item description - Infomação acerca do resultado;
	\item nivel - Se é um Teste/Subteste;
\end{itemize}

\newpage
\centerline{\textbf{Conclusão}}
\vspace{5 em}

Na nossa opinião foi muito interessante o desenvolvimento deste projeto, pois potenciou a experiência do desenvolvimento de Software. Assimilar os conteúdos da Unidade Curricular, desenvolver Capacidades de programação em \textit{PYTHON}, e na linguagem de marcação \textit{HTML}.
\\

\vspace{1 em}
Sentimos que este projeto foi bastante exigente e fez com que nos dedicássemos mais e melhorar-mos as nossas capacidades.
\\

\vspace{1 em}
Com este Trabalhos adquirimos inúmeras valias que nos serão úteis em futuros projetos.
\\

\vspace{1 em}
Em suma, abordamos todos os assuntos lecionados e graças a isso conseguimos cumprir os objetivos propostos.
\\

% ------ Pagina 4
\newpage

\centerline{\textbf{Bibliografia}}
\vspace{3 em}


\end{document}
 
